\documentclass[12pt]{article}
\usepackage[utf8]{inputenc}
\usepackage{amsmath, amssymb, amsthm}
\usepackage{geometry}
\geometry{letterpaper, margin=1in}
\usepackage{graphicx}
\usepackage{hyperref}
\usepackage{cite}

\title{Teleparallel Gravity: Relativistic Gravity in Flat Spacetime}
\author{}
\date{}

\begin{document}

\maketitle

\begin{abstract}
This presentation reviews the mathematical and conceptual framework of Teleparallel Gravity, a theory of gravity that is equivalent to General Relativity. In Teleparallelism, gravity is encoded in torsion instead of curvature, allowing for a description in terms of flat, Minkowski spacetime. This, in principle, allows for an approach to relativistic gravity that more closely parallels other familiar field theories, such as electromagnetism. We will explore conceptual foundations, highlighting the distinctions between changes of reference frame versus changes of coordinates, while also providing a physically transparent way to understand the theory. We illustrate these ideas with a visualizable model involving the stereographic projection of two-dimensional spherical geometry to a two-dimensional plane.
\end{abstract}

\tableofcontents
\newpage

\section{Introduction}
Teleparallel Gravity (TG) offers an alternative formulation to Einstein's General Relativity (GR), where gravity is not described by the curvature of spacetime but by torsion in a flat, Minkowski background. This section will explore the historical motivations for TG, comparing it with GR, and discussing the implications of a flat spacetime approach.

\subsection{Historical Context}
The development of TG can be traced back to early attempts to unify gravity with other fundamental forces. Discussing the trajectory from Einstein-Cartan theory to the rise of TG as a viable alternative to GR, this subsection will explore why the notion of flat spacetime is appealing in the context of gravitational theory.

\subsection{Teleparallel Gravity vs. General Relativity}
While GR uses the curvature of spacetime to describe gravity, TG utilizes torsion in a flat spacetime. We will delve into the mathematical distinctions, emphasizing how the tetrad fields in TG play a role analogous to the metric tensor in GR.

\subsection{Significance of Flat Spacetime}
Flat spacetime simplifies the conceptual framework for understanding gravity, allowing for a direct comparison with other field theories like electromagnetism. Discussing the theoretical implications, this section will highlight how TG enables new ways of visualizing and computing gravitational effects.

\section{Foundations of Relativistic Mechanics and Inertial Frames}
This section establishes the groundwork in relativistic mechanics, starting with the first law of motion and the concept of inertial frames.

\subsection{Inertial and Non-Inertial Frames}
An inertial frame is one where objects remain at rest or in uniform motion unless acted upon by a force. Non-inertial frames involve accelerations. We will discuss the mathematical transformations between these frames using Lorentz transformations and discuss their significance in both GR and TG.

\subsection{Covariant Derivatives and Tensors}
To describe physical laws in a covariant manner, we introduce the concept of covariant derivatives, which allow us to differentiate tensors in a way that respects the underlying geometric structure of spacetime. The formalism of covariant derivatives in the context of TG will be explored in depth.

\subsection{First Law of Motion and Covariant Description}
Expanding upon the first law, we will explore its implications in relativistic mechanics, particularly how acceleration in a given frame can be described in a coordinate-independent way using covariant derivatives.

\section{Covariant Formalism and Reference Frames}
In this section, the covariant formalism is applied to the study of reference frames, distinguishing between changes of reference frame and coordinate transformations.

\subsection{Reference Frames and Coordinates}
While a change in reference frame involves transitioning between different observers' perspectives, a coordinate transformation merely re-labels points in spacetime. We will distinguish these concepts mathematically and discuss their physical implications in both GR and TG.

\subsection{Lorentz Transformations and Frame Changes}
We discuss how Lorentz transformations serve as the core mathematical tool for changing reference frames in special relativity. We will also consider generalizations to non-inertial frames and discuss how these transformations are interpreted differently in TG compared to GR.

\subsection{Covariant Formalism Applied to Teleparallel Gravity}
Here, we will show how TG's formalism allows for a covariant description of gravity in terms of torsion, contrasting this with the curvature-based description in GR. The equations governing the dynamics of torsion will be derived and discussed.

\section{Teleparallel Gravity and the Analogy with Electromagnetism}
One of the advantages of TG is its closer analogy with electromagnetism, allowing for a more unified field theory approach.

\subsection{Field Equations in TG}
Deriving the field equations for TG, we will compare them with the Maxwell equations of electromagnetism, showing the formal similarities and discussing the physical differences.

\subsection{Force-Free Frames in Gravity and Electromagnetism}
We will explore the conditions under which a frame can be found where the effects of a gravitational field vanish, analogous to how an electromagnetic field can be canceled out in a particular reference frame.

\subsection{Implications for Particle Motion}
In this subsection, we will discuss how the motion of particles is affected in TG, drawing parallels to how particles move in electromagnetic fields. The concept of force-free motion in TG will be analyzed in depth.

\section{Spherical Geometry and Stereographic Projection}
A two-dimensional spherical model provides an intuitive way to visualize concepts in TG.

\subsection{Stereographic Projection: A Mathematical Tool}
We will introduce stereographic projection, a method for mapping a sphere onto a plane, and explore its mathematical properties.

\subsection{Visualizing Teleparallelism on the Sphere}
Using stereographic projection, we will develop a model for understanding torsion in TG, illustrating how gravity can be visualized in flat spacetime.

\subsection{Connecting Spherical and Hyperbolic Geometries to Spacetime}
We will explore how concepts from spherical and hyperbolic geometry can be extended to four-dimensional spacetime, providing deeper insights into the structure of TG.

\section{Weitzenböck Connection and Teleparallelism}
In TG, the Weitzenböck connection replaces the Levi-Civita connection used in GR. This section will delve into the mathematical structure and physical interpretation of the Weitzenböck connection.

\subsection{Tetrads and the Weitzenböck Connection}
We will derive the Weitzenböck connection from the tetrads and discuss its role in TG. A comparison with the Christoffel symbols used in GR will be made to highlight the conceptual shift.

\subsection{Torsion as the Fundamental Quantity}
Torsion, rather than curvature, plays the central role in TG. We will develop the equations governing torsion and discuss how they lead to the field equations of TG.

\subsection{Physical Interpretation and Applications}
The Weitzenböck connection provides a different way of interpreting gravitational phenomena. This section will explore applications of TG, such as in cosmology and the study of gravitational waves.

\section{Conclusion and Future Directions}
In concluding, we will summarize the key points and suggest directions for future research.

\subsection{Summary of Key Ideas}
We will recap the distinctions between TG and GR, the significance of flat spacetime, and the role of torsion in describing gravity.

\subsection{Potential for Unification}
TG offers possibilities for unifying gravity with other forces, a potential that we will explore here.

\subsection{Open Questions and Future Research}
Finally, we will discuss the open questions in TG, such as the need for empirical tests, and suggest avenues for further theoretical development.

\newpage
\bibliographystyle{plain}
\bibliography{references}

\end{document}
