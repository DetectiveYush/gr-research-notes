\documentclass[USletter,11pt]{article}
\usepackage{amsmath,amsfonts,amssymb,geometry}

\geometry{letterpaper, margin=1in}


\title{Teleparallel Gravity: Relativistic Gravity in Flat Spacetime}
\author{}
\date{}

\begin{document}

\maketitle

\begin{abstract}
This presentation reviews the mathematical and conceptual framework of Teleparallel Gravity, a theory of gravity that is equivalent to General Relativity. In Teleparallelism, gravity is encoded in torsion instead of curvature, allowing for a description in terms of flat, Minkowski spacetime. This, in principle, allows for an approach to relativistic gravity that more closely parallels other familiar field theories, such as electromagnetism. We will explore conceptual foundations, highlighting the distinctions between changes of reference frame versus changes of coordinates, while also providing a physically transparent way to understand the theory. We illustrate these ideas with a visualizable model involving the stereographic projection of two-dimensional spherical geometry to a two-dimensional plane.
\end{abstract}

\section{Introduction}
Teleparallel Gravity (TG) is a reformulation of General Relativity (GR) that describes gravity using torsion instead of curvature. In TG, spacetime remains flat and torsion becomes the key geometrical object encoding gravitational interactions. This offers an approach to gravity that closely parallels the treatment of other fields, such as electromagnetism, within the flat Minkowski spacetime. This paper aims to explore the mathematical framework of TG, emphasizing its advantages over the traditional curvature-based formulation of GR.

\section{Foundations of Relativistic Mechanics and Inertial Frames}
Relativistic mechanics begins with the notion of inertial frames, defined by the first law of motion as frames where objects move at constant velocity unless acted upon by an external force. In TG, the concept of an inertial frame is retained, but the description of gravitational forces is modified. The exchange between frames and the choice of coordinates becomes critical, and we introduce a covariant derivative that allows us to describe physical phenomena independent of coordinate changes. This covariant formalism is central to understanding TG, where the effects traditionally attributed to gravity in curved spacetime are instead described by torsion in a flat spacetime.

\section{Covariant Formalism and Reference Frames}
In this section, we clarify the distinction between changing coordinates within a given reference frame and changing the reference frame itself. In TG, a change of coordinates does not imply a change in the physical situation, whereas a change in the reference frame (especially to a non-inertial frame) can introduce apparent forces that must be accounted for. The core transformation in this context is the Lorentz transformation, which remains fundamental in TG. However, when transitioning to non-inertial frames, a series of local Lorentz transformations is applied, which can justify local modifications to the torsion field.

\section{Teleparallel Gravity and Electromagnetism: A Comparative Approach}
One of the motivations for TG is the desire to formulate gravity in a manner analogous to electromagnetism. In electromagnetism, the field is described by a vector potential in flat spacetime, and the forces between charged particles can be completely described within this framework. TG seeks to achieve a similar description for gravity, where the gravitational interaction can be described by a torsion field in flat spacetime. However, unlike electromagnetism, gravity has the unique property that there exists a non-inertial frame where the gravitational forces on all particles cancel out. This section explores this analogy in detail and discusses the implications for our understanding of gravitational interactions.

\section{Rindler Coordinates and the Equivalence Principle}
Rindler coordinates provide an example of a reference frame where an observer remains at rest while objects appear to accelerate due to the presence of a uniform gravitational field. This scenario is illustrative in TG because it shows how gravitational effects can be understood in terms of frame-dependent torsion rather than spacetime curvature. The connection to the equivalence principle is made explicit, as TG naturally accommodates the idea that in a local inertial frame (i.e., a freely falling frame), gravitational forces can be completely transformed away. This is a key advantage of the teleparallel approach.

\section{Constant Gravitational Field and the Weitzenböck Connection}
In TG, the Weitzenböck connection replaces the Christoffel symbols of GR. This connection has zero curvature but non-zero torsion, which is responsible for the gravitational interaction. In the presence of a constant gravitational field, the curvature tensor remains zero (indicating flat spacetime), but the torsion tensor is non-zero. This section delves into the mathematical formulation of the Weitzenböck connection and its implications for our understanding of gravity.

\section{Structure Coefficients and Spherical Geometry}
The structure coefficients, which appear in the study of differential geometry, have a direct analogy in TG when considering spherical and hyperbolic geometries. This section explores the relationship between these geometries and spacetime torsion, particularly how the stereographic projection from a sphere to a plane can serve as a model for understanding the effects of torsion in TG. By visualizing these effects, we gain deeper insights into the nature of gravity as described by TG.

\section{Conclusion and Future Directions}
Teleparallel Gravity offers a compelling alternative to the traditional curvature-based description of gravity. By framing gravity in terms of torsion and flat spacetime, TG provides a more physically intuitive understanding of gravitational interactions, drawing parallels with other field theories like electromagnetism. Future work will focus on further developing the mathematical tools needed to fully exploit the potential of TG, including its applications to cosmology and quantum gravity.

\end{document}
