\documentclass{beamer}
\usepackage{amsmath}
\usepackage{amsfonts}
\usepackage{graphicx}
\usepackage{tikz}
\usepackage{hyperref}

% Title Page
\title{Teleparallel Gravity: A Visual and Conceptual Framework}
\subtitle{Equivalence with General Relativity and Alternative Perspectives}
\author{Your Name}
\institute{Your Institution}
\date{\today}

\begin{document}

% Slide 1: Title
\begin{frame}
  \titlepage
\end{frame}

% Slide 2: Abstract
\begin{frame}{Abstract}
This presentation reviews the mathematical and conceptual framework of Teleparallel Gravity, a theory of gravity that is equivalent to General Relativity. We explore the distinctions between changes of reference frame versus changes of coordinates and illustrate these ideas using a visual model based on stereographic projection.
\end{frame}

\section{Introduction to Teleparallelism}

% Slide 3: Introduction to Teleparallel Gravity
\begin{frame}{Introduction to Teleparallel Gravity}
  \begin{itemize}
    \item Teleparallel Gravity: An equivalent formulation to General Relativity.
    \item Encodes gravity in terms of torsion, not curvature.
    \item Describes spacetime as flat (Minkowski), with torsion capturing gravitational effects.
    \item Provides an alternative, field-theoretic perspective on gravity.
  \end{itemize}
\end{frame}

% Slide 4: Textbook Definition
\begin{frame}{Textbook Definition}
  \begin{itemize}
    \item Differential Geometry: Vectors as elements of the tangent space.
    \item Connections: Methods for parallel transport in curved spaces.
    \item In General Relativity: Curvature is defined via the Levi-Civita connection.
    \item In Teleparallel Gravity: Curvature is replaced by torsion using the Weitzenböck connection.
  \end{itemize}
\end{frame}

% Slide 5: Vectors and Torsion in Teleparallelism
\begin{frame}{Vectors and Torsion}
  \begin{itemize}
    \item Torsion describes how a vector field twists and turns.
    \item Tetrad fields provide an orthonormal basis to measure torsion.
    \item Torsion represents the "twisting" of spacetime.
    \item Visualize: Use graphics to illustrate torsion versus curvature.
  \end{itemize}
\end{frame}

\section{Physically Transparent and Intuitive Descriptions}

% Slide 6: Newton's Second Law in Special Relativity
\begin{frame}{Newton's Second Law in Special Relativity}
  \begin{itemize}
    \item Newton's 2nd Law and force connection \( F^a_{bc} \).
    \item Relating force to coordinate changes: introduces connection terms.
    \item For gravity: inertial mass = gravitational mass $\Rightarrow$ same acceleration for all masses.
  \end{itemize}
\end{frame}

% Slide 7: Interpreting Gravity in Teleparallelism
\begin{frame}{Interpreting Gravity in Teleparallelism}
  \begin{itemize}
    \item Start with flat spacetime and match force laws to curved space concepts.
    \item Use torsion to reinterpret acceleration as an effect of spacetime geometry.
    \item Presents a way to visualize and understand gravity differently.
  \end{itemize}
\end{frame}

\section{Distinction Between Coordinate Systems and Reference Frames}

% Slide 8: Changes of Coordinates vs. Reference Frames
\begin{frame}{Changes of Coordinates vs. Reference Frames}
  \begin{itemize}
    \item Infinite choices of coordinate systems without changing the reference frame.
    \item Difference between coordinates change and moving to an accelerated frame.
    \item Reference frame change mimics coordinate changes but involves physical effects (e.g., torsion).
  \end{itemize}
\end{frame}

% Slide 9: Handling Non-Inertial Frames
\begin{frame}{Handling Non-Inertial Frames}
  \begin{itemize}
    \item Structure of terms arising from non-inertial frames.
    \item Torsion effects mimic changes similar to non-inertial frames in teleparallel gravity.
    \item Teleparallel Equivalent of General Relativity (TEGR): Encodes effects through torsion.
  \end{itemize}
\end{frame}

\section{Visualizing with Stereographic Projection}

% Slide 10: Visualizing with Stereographic Projection
\begin{frame}{Visualizing with Stereographic Projection}
  \begin{itemize}
    \item Introduction to stereographic projection: Mapping a sphere to a plane.
    \item Visualize torsion as twisting paths on the plane.
    \item Curved paths represent torsional effects.
  \end{itemize}
\end{frame}

% Slide 11: Implementation and Visualization
\begin{frame}{Implementation and Visualization}
  \begin{itemize}
    \item Graphical demonstration of geodesic and non-geodesic paths.
    \item Show arcs/curves representing torsion versus curvature.
    \item Highlight the differences: This is not spacetime, but an analogy.
  \end{itemize}
\end{frame}

% Slide 12: Conclusion
\begin{frame}{Conclusion}
  \begin{itemize}
    \item Teleparallel Gravity: Provides alternative perspectives on gravity.
    \item Torsion as a different tool to understand spacetime.
    \item Helps in visualizing complex geometrical concepts in an intuitive way.
  \end{itemize}
\end{frame}

% Slide 13: Q&A
\begin{frame}{Questions and Answers}
  \begin{center}
    \textbf{Any questions?}
  \end{center}
\end{frame}

\end{document}
