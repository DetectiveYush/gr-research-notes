\documentclass{beamer}

\usepackage{amsmath}
\usepackage{amsfonts}
\usepackage{amssymb}

\title{Understanding Teleparallel Gravity}
\author{Your Name}
\date{\today}

\begin{document}

\frame{\titlepage}

\section{Introduction to Teleparallelism}
\begin{frame}
  \frametitle{Introduction to Teleparallelism}
  Teleparallel Gravity is an alternative formulation of gravity where the curvature of spacetime is replaced by torsion. In this framework:
  \begin{itemize}
    \item Gravity is encoded in torsion rather than curvature.
    \item We work within flat Minkowski spacetime.
    \item This simplifies the mathematical treatment by using a torsion-based description of gravity.
  \end{itemize}
\end{frame}

\section{Mathematical Formulation}
\begin{frame}
  \frametitle{Levi-Civita Connection}
  The traditional Levi-Civita connection in General Relativity is given by:
  \[
  \Gamma^\lambda_{\mu \nu} = \frac{1}{2} g^{\lambda \sigma} \left( \frac{\partial g_{\sigma \mu}}{\partial x^\nu} + \frac{\partial g_{\sigma \nu}}{\partial x^\mu} - \frac{\partial g_{\mu \nu}}{\partial x^\sigma} \right)
  \]
  where:
  \begin{itemize}
    \item $g_{\mu \nu}$ is the metric tensor.
    \item $g^{\lambda \sigma}$ is the inverse metric tensor.
    \item $\frac{\partial g_{\sigma \mu}}{\partial x^\nu}$ are the partial derivatives of the metric components.
  \end{itemize}
\end{frame}

\begin{frame}
  \frametitle{Weitzenböck Connection}
  In Teleparallel Gravity, we use the Weitzenböck connection, which involves torsion and is defined as:
  \[
  \overset{\star}{\Gamma}^\lambda_{\mu \nu} = \frac{1}{2} \left( \overset{\star}{T}^\lambda_{\mu \nu} + \overset{\star}{T}^\lambda_{\nu \mu} - \overset{\star}{T}_{\mu \nu}^{\phantom{\mu \nu}\lambda} \right)
  \]
  where $\overset{\star}{T}^\lambda_{\mu \nu}$ represents the torsion tensor components.
\end{frame}

\begin{frame}
  \frametitle{Connecting Levi-Civita and Weitzenböck Connections}
  The Levi-Civita connection ($\Gamma^\lambda_{\mu \nu}$) involves curvature and is used in General Relativity. In contrast, the Weitzenböck connection ($\overset{\star}{\Gamma}^\lambda_{\mu \nu}$) is used in Teleparallel Gravity and involves torsion.

  Specifically, the contortion tensor, which relates these connections, is given by:
  \[
  K^\rho_{\mu \nu} = \frac{1}{2} \left( T^\rho_{\nu \mu} + T^\rho_{\mu \nu} - T_{\mu \nu}^\rho \right)
  \]
  where:
  \begin{itemize}
    \item $K^\rho_{\mu \nu}$ is the contortion tensor.
    \item $T^\rho_{\mu \nu}$ represents the components of the torsion tensor.
  \end{itemize}
  This formula is cited from Page 9 of Aldrovandi and Pereira’s book \cite{AldrovandiPereira}.
\end{frame}

\section{Physical Interpretation}
\begin{frame}
  \frametitle{Physical Interpretation}
  \begin{itemize}
    \item In General Relativity, the Christoffel symbols describe how vectors change in curved spacetime.
    \item In Teleparallel Gravity, the Weitzenböck connection in flat spacetime describes gravitational effects using torsion.
    \item This formulation simplifies the equations and helps to visualize gravity as a property of flat spacetime with torsion.
  \end{itemize}
\end{frame}

\section{Visualizing with Stereographic Projection}
\begin{frame}
  \frametitle{Stereographic Projection}
  Stereographic projection can be used to illustrate the effects of torsion:
  \begin{itemize}
    \item This projection maps a spherical surface to a plane, helping to demonstrate the difference between curvature and torsion.
    \item Geodesic and non-geodesic paths on the sphere can represent the effects of torsion, illustrating how torsion affects trajectories differently compared to curvature.
  \end{itemize}
\end{frame}

\begin{thebibliography}{10}
  \bibitem{AldrovandiPereira}
    R. Aldrovandi and J. G. Pereira, \textit{An Introduction to Teleparallel Gravity}, pg. 9.
\end{thebibliography}

\end{document}
